\documentclass[12pt]{article}
\usepackage{geometry} % see geometry.pdf on how to lay out the page. There's lots.
\geometry{letterpaper} 
\usepackage{hanging}
\usepackage{parskip}
\usepackage{fancyhdr}
\pagestyle{fancy}
\usepackage{hyperref}
\usepackage{xltxtra,fontspec}
\setmainfont[Mapping=tex-text]{Hoefler Text}
\usepackage{fourier-orns}
\hypersetup{urlcolor=cyan}
\usepackage{marvosym}
\usepackage{multirow}
\usepackage{pifont}
%\usepackage{draftwatermark}

\setlength{\parindent}{0in}


\fancyhead[LE,LO]{\small \textsc{FYSM 1506R Sociology of the Weird and the Apocalyptic \hfill Fall 2012 / Winter 2013}}
\fancyfoot[CE,CO]{\thepage}

\renewcommand{\headrule}{\vbox to 0pt{\hbox to\headwidth{\hrulefill \hspace{.125in} \bomb \hspace{.125in} \hrulefill} \vss}}


\fancypagestyle{plain}{%
\fancyhf{} % clear all header and footer fields
\fancyfoot[C]{\thepage}
\renewcommand{\headrule}{\vbox to 0pt{\hbox to\headwidth{\hrulefill \hspace{.125in} \bomb \hspace{.125in} \hrulefill} \vss}}}

\setlength{\headsep}{25pt}
\setlength{\headheight}{25pt}

\begin{document}

\thispagestyle{plain}

{\Large \textsc{FYSM 1506R Topics in the Study of Societies}}

{\large Sociology of the Weird and the Apocalyptic}

\textit{Part of the ArtsOne ``Monsters and Monstrosity'' Cluster}

\vspace{.125in}

Fall 2012 / Winter 2013

\vspace{.125in}

Craig McFarlane \\
\href{mailto:craig\_mcfarlane@carleton.ca}{craig\_mcfarlane@carleton.ca} \\
\href{http://www.monstrosity.ca}{http://www.monstrosity.ca}

\vspace{.25in}

{\Large \textsc{Course Description}}

This seminar explores the idea of ``social science fiction'' through contemporary sociology and sociological theory. The emphasis in this course will be on apocalyptic literature and weird tales and how these can be understood as sociological reflections on past guilt, present anxieties and future worries. The word ``apocalypse'' comes from the Greek for ``un-covering'' and is translated into Latin as ``revelation.'' Hence an apocalypse isn't \textit{just} the end of the world: it is an end that reveals something important, such as a secret. The question, then, is what secret is the contemporary apocalyptic imaginary trying to reveal and uncover? The course culminates in a collaborative, collective, and online final project where students, drawing upon social science research and fiction, will develop plans on how humanity can prepare for and respond to a variety of catastrophic and apocalyptic scenarios. Accordingly, we will examine a number of versions of the apocalyptic imaginary ranging from viral pandemics to zombies and vampires. The fall semester will be largely led by the instructor while the winter semester will largely be driven by student interests with small groups of students designing a class each. There will also be a number of activities done in conjunction with the other seminars in the ``Monsters and Monstrosity'' cluster.

Students are advised that as a seminar, this course is intended to be intensive in terms of the workload, including but not limited to the readings, discussions, and assignments. Students are also advised that given the subject matter, this course will take up issues that are unpleasant, horrific, disturbing and violent.

\vspace{.125in}

{\Large \textsc{Course Objectives}}

\begin{itemize}
\item To introduce students to the relation between sociological concepts and the analysis of culture.
\item To encourage students to view culture as more than just mere ``entertainment'' and as a reflection and commentary on on society.
\item To improve the student's ability to write clearly, read critically, and work collaboratively.
\item To familiarize students with increasingly important online collaborative programs such as Wordpress and MediaWiki.
\item To induct the student into the norms of university life.
\end{itemize}

{\Large \textsc{Required Texts}}

The following texts are \textbf{required} for the Fall 2012 semester:

\begin{hangparas}{.5in}{1}
Atwood, Margaret. \textit{Oryx \& Crake.} Toronto: Vintage Canada, 2009. 978-0307398482

Guterl, Fred. \textit{The Fate of the Species: Why the Human Race May Cause Its Own Extinction and How We Can Stop It.} New York: Bloomsbury US, 2012. 978-1608192588

James, P.D. \textit{The Children of Men.} Toronto: Vintage Canada, 2005. 978-0676977691

Lovecraft, H.P. \textit{At the Mountains of Madness.} New York: The Modern Library, 2005. 978-0812974416

Mi\'eville, China. \textit{The City \& The City.} New York: Del Ray, 2010. 978-0345497529

The following texts are \textbf{required} for the Winter 2013 semester:

Kirkman, Robert. \textit{The Walking Dead, Book One.} Berkeley: Image, 2010. 978-1582406190

Whitehead, Colson. \textit{Zone One: A Novel.} New York: Doubleday, 2011. 978-0385528078

\end{hangparas}

All required texts are available for purchase from Octopus Books located at 116 Third Avenue (off Bank Street in The Glebe). All other readings are available on reserve in the library, through cuLearn, or available online.

{\Large \textsc{Course Requirements and Evaluation}}

Unless otherwise indicated, all assignments are due at the start of class the date they are due. Any assignments submitted after the start of class or to the drop box will be deemed late. Late assignments are penalized one grade point per day late (e.g., an assignment two days late which merits a grade of A- will be given a grade of B). Extentions will not be granted under any circumstance unless a formal deferral has been approved by the Registrar. Please note all assignments must be completed in order to pass this course; i.e., failure to complete all assignments will result in a mark of FND. See the final page for an overview of the meaning of the various grades. Plagiarism (see below) will not be tolerated and will result in the matter being referred to the Dean of the Faculty of Arts and Social Sciences and will most likely result in a failure on the assignment, if not also the course. There are no exceptions to any of these policies. While all grades are subject to approval by the Chair of the Department of Sociology and Anthropology and the Dean of the Faculty of Arts and Social Sciences, provisional marks will be posted to cuLearn as they become available.

Evaluation is based upon the following components:

\begin{tabbing}
\hspace{.5in} \= Essay \hspace{1.25in} \= 15\% \\
\> Peer Evaluation \> 5\% \\
\> Weekly Contributions \> 20\% \\
\> Interview \> 10\% \\
\> Final Project \> 30\% \\
\> ArtsOne \> 20\%
\end{tabbing}

{\large \textsc{Essay 15\%} Due November 28}

The short essay is to be written on a topic of the student's own choosing. It should be related to course material from the first semester and should present an analysis and/or interpretation of at least two separate sets of readings from the first semester. The essay is intended to be short---roughly 1500 to 1750 words---and is \textit{due at the start of class on November 28.} Students are required to submit \textit{three} hard copies of the assignment because it will in part be evaluated by peers (see next assignment: Peer Evaluation) and in part be evaluated by the instructor. Your name should not appear on the assignment: \textit{just} your student number---this will prevent your peers from playing favourites and ensure that peer evaluation is based upon the quality of the work rather than on who wrote it. The grade will be an average of the marks assigned by the two student evaluators and the instructor: e.g., first student evaluator assigns a grade of 72\%, second student evaluator assigns a grade of 69\% and the instructor assigns a grade of 71\%:
\[
72\% + 69\% + 71\% = 211\%
\]
\[
{\frac{221\%}{3}}=70.3\%
\]
Thus, the student would receive a grade of 70.3\% or 10.5/15.

{\large \textsc{Peer Evaluation 5\%} Due December 12}

Each student will read two essays written by their peers and will provide constructive feedback to that student and recommend (and justify) a grade on that assignment. The essay should be evaluated on the basis of how you, yourself, would want to be evaluated. Ideally, this means that the content of the essay will be the focus rather than the mechanics of the essay. Hence, you will provide feedback on the quality of the argument, the detail of the analysis, and so on. Feedback does not necessarily mean criticism, but rather constructive engagement with the work of your peers, an indication of what the essay did well, what the essay did not do well, why this is the case, alternative analyses, and so on. The feedback will be submitted to \textit{both} the student you are evaluating \textit{and} to the instructor. Unlike the essay you are evaluating, which is submitted anonymously, you must put your name on the evaluation. Peer evaluations are to be emailed to the instructor \textit{no later} than 11:59PM on December 12. The evaluations \textit{must} be sent as an attachment in PDF \textit{only}. The instructor will then re-distribute the evaluations. \textit{Remember: your name must be on the evaluation.}

{\large \textsc{Weekly Contributions 20\%}}

A seminar is a collective endeavour. It only works to the extent that all members of the seminar actively participate in its intellectual life. There are a number of facets to this: doing the assigned readings, attending class regularly, arriving to class prepared to discuss the readings, participating in class discussions, and so on. 

A website for the course is found at \href{http://www.monstrosity.ca}{http://www.monstrosity.ca}. Wordpress (a program used to host blogs) and MediaWiki (the program used by Wikipedia) have been installed on the site. Accounts will be set up during the first class. Wordpress and MediaWiki will be demonstrated in class, but tutorials and documentation for Wordpress can be found at \href{http://codex.wordpress.org/WordPress\_Lessons}{http://codex.wordpress.org/WordPress\_Lessons} and the same for MediaWiki can be found at \href{http://en.wikipedia.org/wiki/Help:Contents/Getting\_started}{http://en.wikipedia.org/wiki/Help:Contents/Getting\_started}.

In addition to active participation in the classroom, it is expected that students will also actively participate on the course website. Accordingly, the following is expected:

\begin{itemize}
\item Students will make at least six posts to the course blog each semester (i.e., roughly one every two weeks) and posts should be in the neighbourhood of 300-500 words long. The content of the post is to be determined by the student, but it can, for instance, include a reflection on the readings and discussion. In addition to the six substantive posts per semester, students are also invited to make as many shorter posts containing links to articles or sites of interest as they want. (Note: merely linking to the site or article is not sufficient; students should say \textit{why} it is interesting and worth looking at.) \textit{When making your posts to the blog, please make sure that you use the ``FYSM 1506'' category.}
\item It is expected that students will actively check the website, read new posts by other students, and comment on those posts.
\item Students should also contribute to the course wiki, which will be used for two purposes: first, to keep a record of what we have studied (i.e., outlines and analyses of what we've read, key concepts, and so on) and, second, as the basis for the final project (see below).
\end{itemize}

{\large \textsc{Interview 10\%}}

In the second semester, small groups of students (about two or three) will prepare a set of interview questions and find an interview subject who has professional or academic expertise related to monsters, monstrosity, the weird, catastrophes, the apocalyptic, and the like. This could include authors, public intellectuals, professors, government experts, or NGO employees. For instance, students might attempt to interview one of the authors read in class, a public health official on a city's mass fatality plan, or a government emergency planning agency. Preliminary questions must be vetted by the instructor prior to doing the interview. The instructor will assist (as needed) in setting up the interviews. Students should aim to conduct a twenty to thirty minute long interview, which will be recorded (audio or video), transcribed, and posted to the course website as either a YouTube video or podcast. The timeline for completing this assignment is up to the students and the availability of their interview subject. However, recording the interview should be completed by the end of February.

{\large \textsc{ArtsOne 20\%}}

There are two major assignments that will be done in conjunction with the rest of the Cluster. 

\textit{Monster Ethnography} (15\%) In the fall, small groups of students will do a ``monster ethnography'' on Halloween evening, which is an ideal time to observe monsters in their natural environment. The idea of ethnography and an introduction to interview techniques will be discussed in a workshop on September 26. But the essence of the idea is to observe monsters being monstrous on the scariest night of the year. Areas of focus might include the sort of costumes people choose to wear, how the costumes affect behaviour (or not), what sort of monsters tend to congregate with one another, what sort of monsters tend not to congregate with one another, and so on. The general rule is that research subjects must be disguised as monsters (and not as celebrities or wearing skimpy clothing and rabbit ears) and they must be out on the street or otherwise at an assembly of monsters. Students might approach the assignment through the methods of participant observation, interviews, or otherwise. Students should prepare to record video and/or audio of their interviews.

Groups will be assembled by September 26. A draft of proposed questions will be submitted to the instructor on October 10 and returned to the students on October 17. The final draft of questions will be ready on October 31. The cluster will meet as a whole on October 31 to finalize preparations in advance of the fieldwork portion of the assignment. Each seminar will report back on their research to the entire cluster, scheduled for November 7, November 14 (our class will present this day), and November 21. The presentations are intended to be brief (about an hour for each seminar) and will present an analysis of the research data: e.g., where you conducted your research, the questions that you asked, demographic data, the responses you received, an analysis of the responses. Students should also prepare to present some of the recordings they did. Each group will submit a brief written report (roughly 1500 words) the day of their presentation.

\textit{Works in Progress Conference} (5\%) In the winter, small groups of students will present to the entire cluster on aspects of their work relating to the final project. Each presentation will be short, only about five to ten minutes long, and serves to, on the one hand, help you identify the core of what you are working on and, on the other hand, give you an opportunity to tell other students what you are working on and get their feedback on it. This event will be held on February 27. Each seminar will have roughly an hour to do their presentations.

{\large \textsc{Final Project 20\%}}

The final project will be a collaborative class effort to design a plan for reconstructing society---if such a goal is desirable in the first place--in the wake of a cataclysmic or apocalyptic event. The final project will be presented in the form of a Wiki. There are two aspects to this. First, the general plan that would be applicable in nearly all events of this sort. Second, specific plans that are relative to specific scenarios: for instance, natural disaster (massive floods, earthquakes, fires, etc), ecological fall-out (global warming, rising sea levels, disappearance of potable water, etc), biological (pandemic virus), technological fall-out (technology stops working or fuel sources run out completely), rise of the machines or alien invasion, the walking dead (or other biohazard), war and/or terrorism. 

Given that such events are comparatively rare in the history of humanity---perhaps even non-existent---it is necessary to, on the one hand, draw upon existing sociological knowledge about how societies form, function and disintegrate and, on the other hand, imaginary depictions of these events. Hence, fantasy, horror, and science fiction films, novels, and television shows will provide much of the information we have about these events which should be supplemented with what we know about catastrophic disasters---natural disasters such as Hurricane Katrina, the Japanese tsunami, or the Great Slave Lake fire, environmental disasters such as Chernobyl, economic disasters, or the complete destruction of social infrastructure following war---and how societies have attempted to rebuild themselves (successfully or not) in the face of this widespread destruction.

The result of this is that a plan must be made to deal with that which cannot be planned for. The plan must take into account how different scenarios present different problems in particular, but also how each of these different scenarios can, in many ways, be understood in the context of a general plan for dealing with social disintegration. Thus, it is necessary to plan not only for apocalyptic scenarios that come from an external force (zombies, climate change, natural disaster, alien invasion), but also catastrophic scenarios that arise from an internal force (extreme recession, failure of political legitimacy, civil war). As an added challenge, because there will be no surviving structures of authority \textit{after} the event, it is impossible to have recourse to ``the law.'' It will therefore be necessary to identify regulative norms that are not reducible to legal authority or political legitimacy.

Because we are planning for the future and because we assume that the internet is sufficiently indestructible that it will survive most apocalyptic scenarios, the assignment will be done online through the course Wiki.

{\large \textsc{Grades}}

In accordance with the Carleton University Undergraduate Calendar (page 45), the letter grades assigned in this course will have the following percentage equivalents:

\begin{tabbing}
A+ \hspace{.125in} \= 90--100 \hspace{.25in} \= B+ \hspace{.125in} \= 77--79 \hspace{.25in} \= C \hspace{.125in} \= 67--69 \hspace{.25in} \= D+ \hspace{.125in} \= 57--59 \\
A \> 85--89 \> B \> 73--76 \> C \> 63--66 \> D \> 53--56 \\
A- \> 80--84 \> B- \> 70--72 \> C- \> 60--62 \> D- \> 50--52 \\
ABS \> Student absent from final exam \\
DEF \> Deferred \\
FND \> Student could not pass the course even with 100\% on the final exam
\end{tabbing}

For more information on the meaning of grades, see the final page of this syllabus.

\newpage

{\large \textsc{Please Note}}

This course is a seminar. Seminars differ from lectures in that they pre-suppose active participation from all students and the instructor serves the role as facilitator. Accordingly, it is \textit{expected} that you arrive in class prepared to discuss the assigned material. \textit{If you cannot be bothered to meet this minimum requirement, then you have failed in your basic duty as a student and it is best that you don't bother attending class.} It is your right to do or not do your work as you please, but it is \textit{not} your right to ruin the pedagogical experience of others by failing to be adequately prepared.

In order to facilitate participation---and minimize distraction---computers \emph{will not} be permitted in the classroom (unless the use thereof is an accommodation approved by the Paul Menton Centre or unless a computer is needed to complete in-class assignments). Likewise, texting or any other use of cell phones, iPads, and the like will not be tolerated. If you insist on texting or otherwise fooling around on a ``smartphone,'' tablet or computer, you will be asked to leave. If you'd rather watch YouTube videos or chat on Facebook, you might as well stay home---the seating will be more comfortable and the internet connection better.

It is also expected that students are judicious in their use of email. Hence, when contacting the instructor via email, it is expected that you will use your Carleton email account (this is a legal requirement), put the course code and a brief description of the email in the subject line, and write the body of your email in coherent English (i.e., full sentences, proper spelling, grammar and punctuation). If you can't be bothered to write a proper email, I cannot be bothered to reply---after all, the email is clearly not important to you!

Finally, I cannot emphasize strongly enough how important it is to keep up with assigned readings and to attend all the classes. The material is intentionally difficult and challenging. It is your responsibility to show up ready to learn; it is my job to help you meet your responsibility.

{\Large \textsc{Academic Regulations, Accommodations, \& Plagiarism}}

University rules regarding registration, withdrawal, appealing marks, and most anything else you might need to know can be found on \href{http://www.carleton.ca/cu0708uc/regulations/acadregsuniv.html}{the university's website}.

{\large \textsc{Requests for Academic Accommodations}}

\textit{For Students with Disabilities} The Paul Menton Centre for Students with Disabilities (PMC) provides services to students with Learning Disabilities (LD), psychiatric/mental health disabilities, Attention Deficit Hyperactivity Disorder (ADHD), Autism Spectrum Disorders (ASD), chronic medical conditions, and impairments in mobility, hearing, and vision. If you have a disability requiring academic accommodations in this course, please contact PMC at 613-520-6608 or \href{mailto:pmc@carleton.ca}{pmc@carleton.ca} for a formal evaluation. If you are already registered with the PMC, contact your PMC coordinator to send me your Letter of Accommodation at the beginning of the term, and no later than two weeks before the first in-class scheduled test or exam requiring accommodation (if applicable). After requesting accommodation from PMC, meet with me to ensure accommodation arrangements are made. Please consult the PMC website for the deadline to request accommodations for the formally-scheduled exam (if applicable).

\textit{For Religious Obligations} Students requesting academic accommodation on the basis of religious obligation should make a formal, written request to their instructors for alternate dates and/or means of satisfying academic requirements. Such requests should be made during the first two weeks of class, or as soon as possible after the need for accommodation is known to exist, but no later than two weeks before the compulsory event. Accommodation is to be worked out directly and on an individual basis between the student and the instructor(s) involved. Instructors will make accommodations in a way that avoids academic disadvantage to the student. Students or instructors who have questions or want to confirm accommodation eligibility of a religious event or practice may refer to the Equity Services website for a list of holy days and Carleton's Academic Accommodation policies, or may contact an Equity Services Advisor in the Equity Services Department for assistance.

\textit{For Pregnancy} Pregnant students requiring academic accommodations are encouraged to contact an Equity Advisor in Equity Services to complete a letter of accommodation. The student must then make an appointment to discuss her needs with the instructor at least two weeks prior to the first academic event in which it is anticipated the accommodation will be required.

\vspace{.125in}

{\large \textsc{Plagiarism}}

Plagiarism is the passing off of someone else's work as your own and is a serious academic offence. For the details of what constitutes plagiarism, the potential penalties and the procedures refer to the section on Instructional Offences in the Undergraduate Calendar.

\textit{What are the Penalties for Plagiarism?} A student found to have plagiarized an assignment may be subject to one of several penalties including: expulsion; suspension from all studies at Carleton; suspension from full-time studies; and/or a reprimand; a refusal of permission to continue or to register in a specific degree program; academic probation; award of an FND, Fail, or an ABS.

\textit{What are the Procedures?} All allegations of plagiarism are reported to the Dean of the Faculty of Arts and Social Sciences. Documentation is prepared by instructors and/or departmental chairs. The Dean writes to the student and the University Ombudsperson about the alleged plagiarism. The Dean reviews the allegation. If it is not resolved at this level then it is referred to a tribunal appointed by the Senate. 

Students are expected to familiarize themselves with and follow the \href{http://www2.carleton.ca/studentaffairs/academic-integrity}{Carleton University Student Academic Integrity Policy}. The Policy is strictly enforced and is binding on all students.

{\large \textsc{Assistance for Students}}

\href{http://www2.carleton.ca/sasc/}{\textbf{Student Academic Success Centre (SASC)} http://www2.carleton.ca/sasc/}

\href{http://www2.carleton.ca/sasc/writing-tutorial-service/}{\textbf{Writing Tutorial Services} http://www2.carleton.ca/sasc/writing-tutorial-service/} 

\href{http://www1.carleton.ca/sasc/peer-assisted-study-sessions/}{\textbf{Peer Assisted Study Sessions (PASS)} http://www1.carleton.ca/sasc/peer-assisted-study-sessions/}

\newpage

\begin{center}
{\Large \textsc{Course Schedule}}

\vspace{.125in}

{\Large \textsc{Fall 2012}}
\end{center}

\vspace{.25in}

\textbf{September 12, 2012}

\begin{hangparas}{.5in}{1}
Bostrom, Nick and Milan M. \'Cirkovi\'c. ``Introduction.'' In \textit{Global Catastrophic Risks,} edited by Nick Bostrom and Milan M. \'Cirkovi\'C, 1--29. Oxford: Oxford UP, 2008.

Cohen, Jeffrey Jerome. ``Monster Culture (Seven Theses).'' In \textit{Monster Theory: Reading Culture,} edited by Jeffrey Jerome Cohen, 3--25. Minneapolis: University of Minnesota Press, 1996. \href{http://bit.ly/NNmn4Y}{\textbf{http://bit.ly/NNmn4Y}}

Gilmore, David D. \textit{Monsters: Evil Beings, Mythical Beasts, and all Manner of Imaginary Terrors,} 1--22. Philadelphia: University of Pennsylvania Press, 2003.

Hall, John R. \textit{Apocalypse: From Antiquity to the Empire of Modernity.} Cambridge: Polity, 2009. [\textbf{Chapters 1 and 7}]

Hanson, Robin. ``Catastrophe, Social Collapse, and Human Extinction." In \textit{Global Catastrophic Risks,} edited by Nick Bostrom and Milan M. \'Cirkovi\'C, 363--77. Oxford: Oxford UP, 2008.
\end{hangparas}

Note: you do not need to read all of these for this class, but you should read them sooner rather than later because they are important background materials.

\textbf{September 19, 2012}

\begin{hangparas}{.5in}{1}
Mi\'eville, China. ``Introduction.'' In \textit{At the Mountains of Madness,} xi--xxv. New York: The Modern Library, 2005.

Lovecraft, H.P. \textit{At the Mountains of Madness.} New York: The Modern Library, 2005.

Recommended: Joshi, S.T. \textit{The Weird Tale,} 1--11. Austin: University of Texas Press, 1990. \href{http://bit.ly/P5GoDl}{\textbf{http://bit.ly/P5GoDl}}
\end{hangparas}

\textbf{September 26, 2012}

Ethnographic methods workshop with Samah Sabra. 

\begin{hangparas}{.5in}{1}
Clair, Robin. ``Reflexivity and Rhetorical Ethnography: From Family Farm to Orphanage and Back Again.'' \textit{Cultural Studies} \ding{214} \textit{Critical Methodologies} 11, no. 2 (2011): 117-28. \href{http://bit.ly/Q6h9G6}{\textbf{http://bit.ly/Q6h9G6}}
\end{hangparas}

\textbf{October 3, 2012}

\begin{hangparas}{.5in}{1}
Mi\'eville, China. \textit{The City \& The City.} New York: Del Ray, 2010. [\textbf{Part 1}]

Durkheim, Emile. \textit{Elementary Forms of Religious Life,} 8--18, 33--9. Translated by Karen E. Fields. New York: The Free Press, 1995.

Recommended: VanderMeer, Jeff. ``The New Weird---`It's Alive?''' In \textit{Monstrous Creatures: Explorations of Fantasy Through Essays, Articles and Reviews,} 46--54. Bowie, MD: Guide Dog Books, 2011.
\end{hangparas}

\textbf{October 10, 2012}

\begin{hangparas}{.5in}{1}
Mi\'eville, China. \textit{The City \& The City.} New York: Del Ray, 2010. [\textbf{Parts 2, 3, and Coda}]

Durkheim, Emile. \textit{Elementary Forms of Religious Life,} 208--16, 303--13, 321--4. Translated by Karen E. Fields. New York: The Free Press, 1995.
\end{hangparas}

\textbf{October 17, 2012}

\begin{hangparas}{.5in}{1}
Atwood, Margaret. \textit{Oryx \& Crake.} Toronto: Vintage Canada, 2009. [\textbf{Parts 1--5}]

Guterl, Fred. \textit{The Fate of the Species: Why the Human Race May Cause Its Own Extinction and How We Can Stop It.} New York: Bloomsbury US, 2012. [\textbf{Introduction, Chapters 1 and 2}]

Recommended: Gerlach, Neil and Sheryl N. Hamilton. ``A History of Social Science Fiction.'' \textit{Science Fiction Studies} 30, no. 2 (2003): 161--173. \href{http://bit.ly/Q6j0e0}{\textbf{http://bit.ly/Q6j0e0}}
\end{hangparas}

\textbf{October 24, 2012}

\begin{hangparas}{.5in}{1}
Atwood, Margaret. \textit{Oryx \& Crake.} Toronto: Vintage Canada, 2009. [\textbf{Parts 6--10}]

Guterl, Fred. \textit{The Fate of the Species: Why the Human Race May Cause Its Own Extinction and How We Can Stop It.} New York: Bloomsbury US, 2012. [\textbf{Chapters 3 and 4}]
\end{hangparas}

Watch ``Halloween'' (dir. John Carpenter, 1978).

\textbf{October 31, 2012 \hspace{.0625in} \Bat \hspace{.0625in} A Very ``Monsters and Monstrosity'' Halloween \hspace{.0625in} \Bat}

Discussion of the film ``Halloween'' and preparation for the monster ethnography assignment.

\newpage

\textbf{November 7, 2012}

\begin{hangparas}{.5in}{1}
Atwood, Margaret. \textit{Oryx \& Crake.} Toronto: Vintage Canada, 2009. [\textbf{Parts 11--15}]

Guterl, Fred. \textit{The Fate of the Species: Why the Human Race May Cause Its Own Extinction and How We Can Stop It.} New York: Bloomsbury US, 2012. [\textbf{Chapters 5 and 6, and ``Ingenuity''}]
\end{hangparas}

First report on monster ethnography assignment (1:30-2:30).

\textbf{November 14, 2012}

\begin{hangparas}{.5in}{1}
James, P.D. \textit{The Children of Men.} Toronto: Vintage Canada, 2005. [\textbf{Part 1}]

Benatar, David. \textit{Better Never to Have Been: The Harm of Coming Into Existence,} 28--49. Oxford: Oxford UP, 2006. \href{http://bit.ly/I3d0gn}{\textbf{http://bit.ly/I3d0gn}}

FYSM 1506 report on and hand in monster ethnography assignment (1:30--2:30).
\end{hangparas}

\textbf{November 21, 2012}

\begin{hangparas}{.5in}{1}
James, P.D. \textit{The Children of Men.} Toronto: Vintage Canada, 2005. [\textbf{Part 2}]

Benatar, David. \textit{Better Never to Have Been: The Harm of Coming Into Existence,} 163--8, 182--200. Oxford: Oxford UP, 2006. \href{http://bit.ly/I3d0gn}{\textbf{http://bit.ly/I3d0gn}}
\end{hangparas}

Final report on monster ethnography assignment (1:30-2:30).

\textbf{November 28, 2012}

Watch ``Children of Men'' (dir. Alfonso Cuar\'on, 2006).

\vspace{.125in}

\begin{center}
{\Large \textsc{Winter 2013}}
\end{center}

\vspace{.125in}


\textbf{January 9, 2013}

\begin{hangparas}{.5in}{1}
Kee, Chera. ```They are not men\ldots they are dead bodies': From Cannibal to Zombie and Back Again.'' In \textit{Better Off Dead: The Evolution of the Zombie as Post-Human,} edited by Deborah Christie and Sarah Juliet Lauro, 9--23. New York: Fordham UP, 2011. 

Deghoul, Frank. ```We are the mirror of your fears': Haitian Identity and Zombification.'' In \textit{Better Off Dead: The Evolution of the Zombie as Post-Human,} edited by Deborah Christie and Sarah Juliet Lauro, 24--38. New York: Fordham UP, 2011. 

Watch ``White Zombie'' (dir. Victor Halperin, 1932).

Watch ``Night of the Living Dead'' (dir. George A. Romero, 1968).
\end{hangparas}

\textbf{January 16, 2013}

\begin{hangparas}{.5in}{1}
Kirkman, Robert. \textit{The Walking Dead, Book One.} Berkeley: Image, 2010.

Butler, Judith. ``Violence, Mourning, Politics.'' In \textit{Precarious Life: The Powers of Mourning and Violence,} 19--49. London: Verso, 2004.
\end{hangparas}

\textbf{January 23, 2013}

\begin{hangparas}{.5in}{1}
Whitehead, Colson. \textit{Zone One: A Novel.} New York: Doubleday, 2011. [\textbf{First half}]

Alexander, Jeffrey C. ``Cultural Trauma: A Social Theory.'' In \textit{Trauma: A Social Theory,} 6--30. Cambridge: Polity, 2012.
\end{hangparas}

\textbf{January 30, 2013}

\begin{hangparas}{.5in}{1}
Andr\'e Loiselle ``The Theatricality of Monsters.''

Loiselle, Andr\'e. ``Cin\'ema du Grand Guignol: Theatricality in the Horror Film.'' In \textit{Stages of Reality: Theatricality in Cinema}, edited by Andr\'e Loiselle and Jeremy Maron, 55--80. Toronto: University of Toronto Press, 2012.

Watch ``Theatre of Blood.''
\end{hangparas}

\textbf{February 6, 2013}

\begin{hangparas}{.5in}{1}
Whitehead, Colson. \textit{Zone One: A Novel.} New York: Doubleday, 2011. [\textbf{Second half}]

Smelser, Neil J. ``September 11, 2001, as Cultural Trauma.'' In \textit{Cultural Trauma and Collective Identity,} 264--82. Berkeley: University of California Press, 2004.

Craig McFarlane, ``\textit{True Blood} and the Politics of Consumption.'' (1:30-2:30).
\end{hangparas}

\textbf{February 13, 2013}

Workshop: Library (location to be announced)

Aalya Ahmad, ``Monstrous Women in Literature and Folklore'' (1:30-2:30).

\textbf{February 20, 2013}

Reading week.

\newpage

\textbf{February 27, 2013}

``Works in Progress'' conference. 

\textbf{March 6, 2013}

Student group seminars.

\textbf{March 13, 2013}

Student group seminars.

\textbf{March 20, 2013}

Work on final projects (location to be announced).

\textbf{March 27, 2013}

Work on final projects (location to be announced).

\textbf{April 3, 2013}

Presentations on final project to the entire cluster.

\newpage

{\footnotesize
\begin{tabular}{|l|l|l|l|}
\hline
\textbf{GPA} & \textbf{\%} & \textbf{Grade} & \textbf{Notes} \\
\hline
\multirow{3}{*}{12} & \multirow{3}{*}{90--100} & \multirow{3}{*}{A+} & \multirow{10}{4in}{\textbf{Excellent.} The assignment demonstrates a thorough understanding of the readings. Comments on materials are perceptive and original and show a comprehension of the material that goes far beyond what is covered in class. Examples are well integrated, relevant, and presented thoughtfully. The essay is \textit{very well written} and free of structural and grammatical errors. The essay is formatted properly and all sources are properly cited. An ``A+'' is very rarely given--this grade is reserved for truly exceptional work. An ``A'' meets \textit{all} these standards. An ``A-'' meets the majority of these standards.} \\ 
& & & \\
& & & \\
\multirow{3}{*}{11} & \multirow{3}{*}{85--89} & \multirow{3}{*}{A} & \\
& & & \\
& & & \\
\multirow{3}{*}{10} & \multirow{3}{*}{80--84} & \multirow{3}{*}{A-} & \\ 
& & & \\
& & & \\
& & & \\
\hline
\multirow{3}{*}{9} & \multirow{3}{*}{77--79} & \multirow{3}{*}{B+} & \multirow{9}{4in}{\textbf{Good.} The assignment demonstrates a good understanding of the readings. Comments on the material are solid without being particularly original. Examples from the material are well integrated and relevant. The is essay \textit{well written} and has only a few structural and grammatical errors. The essay is formatted properly and all sources are properly cited. A ``B+'' meets \textit{all} of these standards. A ``B'' meets a majority of these standards. A ``B-'' meets some of these standards.}\\
& & & \\
& & & \\
\multirow{3}{*}{8} & \multirow{3}{*}{73--76} & \multirow{3}{*}{B} & \\
& & & \\
& & & \\
\multirow{3}{*}{7} & \multirow{3}{*}{70--72} & \multirow{3}{*}{B-} & \\
& & & \\
& & & \\
\hline
\multirow{3}{*}{6} & \multirow{3}{*}{67--69} & \multirow{3}{*}{C+} & \multirow{9}{4in}{\textbf{Adequate.} The assignment presents an adequate comprehension of the material in general, but certain aspects of the material were not clearly understood. Comments on the material are valid but lack insight and originality. Examples from the material are not always convincing and relevant. The essay is competently written but has some structural and grammatical weaknesses, as well as errors in formatting and citations. A ``C+'' meets \textit{all} of these standards. A ``C'' meets a majority of these standards. A ``C-'' meets some of these standards.} \\
& & & \\
& & & \\
\multirow{3}{*}{5} & \multirow{3}{*}{63--66} & \multirow{3}{*}{C} & \\
& & & \\
& & & \\
\multirow{3}{*}{4} & \multirow{3}{*}{60--6} & \multirow{3}{*}{C-} & \\
& & & \\
& & & \\
\hline
\multirow{3}{*}{3} & \multirow{3}{*}{57--59} & \multirow{3}{*}{D+} & \multirow{9}{4in}{\textbf{Poor.} The assignment shows a lack of understanding of many aspects of the materials. Comments on the materials are weak, dealing simplistically with the most obvious aspects of the materials. Examples are not convincing. The essay is not competently written, with several structural and grammatical weaknesses, as well as errors in formatting and citations. Grades in the ``D'' range are near failures. They are used to signal to students that there are \textit{very serious problems} with their work, and that they must greatly improve their performance in order to do adequately in the course.} \\
& & & \\
& & & \\
\multirow{3}{*}{2} & \multirow{3}{*}{53--56} & \multirow{3}{*}{D} & \\
& & & \\
& & & \\
\multirow{3}{*}{1} & \multirow{3}{*}{50--52} & \multirow{3}{*}{D-} & \\
& & & \\
& & & \\
\hline
\multirow{4}{*}{0} & \multirow{4}{*}{0--49} & \multirow{4}{*}{F} & \multirow{4}{4in}{\textbf{Failure.} The essay completely misses the mark and shows no understanding of the material. The essay is incompetently written, full of structural and grammatical problems, as well as errors in formatting and citations.}\\
& & & \\
& & & \\
& & & \\
\hline
\multirow{2}{*}{N/A} & \multirow{2}{*}{N/A} & \multirow{2}{*}{FND} & \multirow{2}{4in}{\textbf{Failure No Deferral.} A major assignment was not submitted and the Registrar has not granted a deferral.}\\
& & & \\
\hline
\multirow{2}{*}{N/A} & \multirow{2}{*}{N/A} & \multirow{2}{*}{GNA} & \multirow{2}{4in}{\textbf{Grade Not Assigned.} Used as a placeholder pending the result of an academic integrity investigation.}\\
& & & \\
\hline
\end{tabular}
}

\end{document}
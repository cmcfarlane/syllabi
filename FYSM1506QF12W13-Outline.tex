\documentclass[12pt]{article}
\usepackage{geometry} % see geometry.pdf on how to lay out the page. There's lots.
\geometry{letterpaper} 
\usepackage{hanging}
\usepackage{parskip}
\usepackage{fancyhdr}
\pagestyle{fancy}
\usepackage{hyperref}
\usepackage{xltxtra,fontspec}
\setmainfont[Mapping=tex-text]{Hoefler Text}
\usepackage{fourier-orns}
\hypersetup{urlcolor=cyan}
\usepackage{marvosym}
\usepackage{multirow}
%\usepackage{draftwatermark}

\setlength{\parindent}{0in}


\fancyhead[LE,LO]{\small \textsc{FYSM 1506Q Power and Violence \hfill Fall 2012 / Winter 2013}}
\fancyfoot[CE,CO]{\thepage}

\renewcommand{\headrule}{\vbox to 0pt{\hbox to\headwidth{\hrulefill \hspace{.125in} \bomb \hspace{.125in} \hrulefill} \vss}}


\fancypagestyle{plain}{%
\fancyhf{} % clear all header and footer fields
\fancyfoot[C]{\thepage}
\renewcommand{\headrule}{\vbox to 0pt{\hbox to\headwidth{\hrulefill \hspace{.125in} \bomb \hspace{.125in} \hrulefill} \vss}}}

\setlength{\headsep}{25pt}
\setlength{\headheight}{25pt}

\begin{document}

\thispagestyle{plain}

{\Large \textsc{FYSM 1506Q Topics in the Study of Societies}}

{\large Power and Violence}

\textit{Part of the ArtsOne ``Criminal Matters'' Cluster}

\vspace{.125in}

Fall 2012 / Winter 2013

\vspace{.125in}

Craig McFarlane \\
\href{mailto:craig\_mcfarlane@carleton.ca}{craig\_mcfarlane@carleton.ca} \\
\href{http://powerandviolence.ca}{http://powerandviolence.ca}

\vspace{.125in}

{\Large \textsc{Course Description}}

This seminar explores the relation between power and violence in modern and non-modern societies from a sociological perspective. The first semester is oriented towards developing theoretical and conceptual tools useful in the analysis of power and violence while the second semester turns to more substantive topics drawing upon movies, television and novels. This course is not about, strictly speaking, crime, but about how power and violence manifest themselves in society, are controlled and used in society, and what our imaginary reflections on power and violence tell us about our own society. Accordingly, the first semester will look at differences between modern and non-modern societies in how they organize violence. The emphasis here will be on increasing control over the individual body through the processes of modernization and civilization. We will pay particular attention to whether there has been a historical trend towards more or less violence and how this can be understood in relation to changing social forms. This entails spending a lot of time talking about barbarians, knights, youth gangs, revolutions, revolts, executions and prisoners. The second semester turns to contemporary anxieties surrounding violence, especially as they are represented in culture. As a result, we will take up torture, murder, war, and zombies. The course culminates in a final project where students will be asked to apply theoretical concepts in an analysis of a popular culture artefact (e.g., movie, television show, comic book, novel, music, etc) that thematizes power and violence.

\vspace{.125in}

{\Large \textsc{Course Objectives}}

\begin{itemize}
\item To introduce the student to reading theoretical texts and understanding theoretical perspectives with the ultimate goal of preparing students to integrate them into future coursework.
\item To show students that power and violence are permanent features of all human societies and that insofar as these are taken to be problems, simple solutions are not possible.
\item To encourage students to view culture as more than just mere ``entertainment'' and as a reflection and commentary on on society.
\item To improve the student's ability to write clearly, read critically, and work collaboratively.
\item To induct the student into the norms of university life.
\end{itemize}

{\Large \textsc{Required Texts}}

The following texts are \textbf{required} for the Fall 2012 semester:

\begin{hangparas}{.5in}{1}

Muchembled, Robert. \textit{A History of Violence: From the End of the Middle Ages to the Present.} Translated by Jean Birrell. Cambridge: Polity, 2012. 978-0745647470

Pinker, Steven. \textit{The Better Angels of Our Nature: Why Violence Has Declined.} New York: Viking, 2011. 978-0670022953

The following texts are \textbf{required} for the Winter 2013 semester:

Kirkman, Robert. \textit{The Walking Dead, Book One.} Berkeley: Image, 2010. 978-1582406190

Mi\'eville, China. \textit{The City \& The City.} New York: Del Ray, 2010. 978-0345497529
\end{hangparas}

All required texts are available for purchase from Octopus Books located at 116 Third Avenue (off Bank Street in The Glebe). All other readings are available on reserve in the library, through cuLearn, or online.

\vspace{.125in}

{\Large \textsc{Course Requirements and Evaluation}}

Unless otherwise indicated, all assignments are due at the start of class the date they are due. Any assignments submitted after the start of class or to the drop box will be deemed late. Late assignments are penalized one grade point per day late (e.g., an assignment two days late which merits a grade of A- will be given a grade of B). Extentions will not be granted under any circumstance unless a formal extention has been approved by the Registrar. Please note all assignments must be completed in order to pass this course; i.e., failure to complete all assignments will result in a mark of FND. Plagiarism (see below) will not be tolerated and will result in the matter being referred to the Dean of the Faculty of Arts and Social Sciences and will most likely result in a failure on the assignment, if not also the course. There are no exceptions to any of these policies. While all grades are subject to approval by the Chair of the Department of Sociology and Anthropology and the Dean of the Faculty of Arts and Social Sciences, provisional marks will be posted to cuLearn as they become available.

Evaluation is based upon the following components:

\begin{tabbing}
\hspace{.5in} \= Two Essays \hspace{1.25in} \= 40\% \hspace{.25in} \= (2 x 20\%)\\
\> Peer Evaluation \> 10\% \> (2 x 5\%) \\
\> Weekly Contributions \> 20\% \\
\> Final Project \> 25\% \\
\> ArtsOne \> 5\%
\end{tabbing}

{\large \textsc{Essays 40\% (2 x 20\%)}}

\hspace{.5in} First Essay Due December 3, 2012

\hspace{.5in} Second Essay Due February 25, 2013

The first essay is a critical evaluation of both Robert Muchembled's \textit{A History of Violence} and Steven Pinker's \textit{The Better Angels of Our Nature.} The essay is intended to be moderately substantial---roughly 2000 words---and is due at the start of class on December 3, 2012. 

The second essay is on a topic of the student's own choosing relating to materials read and discussed in the second semester. This essay is also intended to be moderately substantial---again, roughly 2000 words---and is due at the start of the class on February 25, 2013.

Students are required to submit \textit{three} hard copies of each essay because it will in part be evaluated by your peers (see next assignment: Peer Evaluation) and in part be evaluated by the instructor. Your name should not appear on the assignment: \textit{just} your student number---this will prevent students from playing favourites and ensure that peer evaluation is based upon the quality of the work and not on who wrote it. The grade of the essay will be averaged between the marks assigned by the two student evaluators and the instructor: e.g., first student evaluator assigns a grade of 72\%, second student evaluator assigns a grade of 69\% and the instructor assigns a grade of 71\%:
\[
72\% + 69\% + 71\% = 211\%
\]
\[
{\frac{221\%}{3}}=70.3\%
\]
Thus, the student would receive a grade of 70.3\% or 10.5/15.

{\large \textsc{Peer Evaluation 10\% (2 x 5\%)}}

\hspace{.5in} First Peer Evaluation Due December 12, 2012

\hspace{.5in} Second Peer Evaluation Due March 11, 2013

For each of the two essay assignments, each student will read two essays written by their peers and will provide constructive feedback to that student and recommend (and justify) a grade on that assignment. The essay should be evaluated on the basis of how you, yourself, would want to be evaluated. Ideally, this means that the content of the essay will be the focus rather than the mechanics of the essay. Hence, you will provide feedback on the quality of the argument, the detail of the analysis, and so on. Feedback does not necessarily mean criticism, but rather constructive engagement with the work of your peers, an indication of what the essay did well, what the essay did not do well, why this is the case, alternative analyses, and so on. The feedback will be submitted to \textit{both} the student you are evaluating \textit{and} to the instructor. Unlike the essay you are evaluating, which is submitted anonymously, you must put your name on the evaluation. In other words, you won't know who you are evaluating, but the person you are evaluating will know who you are. The instructor will then re-distribute the evaluations. \textit{Remember: your name must be on the evaluation.}

The first peer evaluation should be emailed to the instructor by the end of the day on December 12, 2012 in either RTF or PDF format. The second peer evaluation should be submitted at the start of class on March 11, 2013 in hard copy. You should bring two copies of each evaluation with you to class on that day.

{\large \textsc{Weekly Contributions 20\%}}

A seminar is a collective endeavour. It only works to the extent that all members of the seminar actively participate in its intellectual life. There are a number of facets to this: doing the assigned readings, attending class regularly, arriving to class prepared to discuss the readings, participating in class discussions, and so on. 

A website for the course is found at \href{http://powerandviolence.ca}{http://powerandviolence.ca}. Wordpress, a program used to host blogs, has been installed on the site. Accounts will be set up during the first class. The basic mechanics of Wordpress will be demonstrated in class, but a tutorial can be found at \href{http://codex.wordpress.org/WordPress\_Lessons}{http://codex.wordpress.org/WordPress\_Lessons}.

In addition to active participation in the classroom, it is expected that students will also actively participate on the course website. Accordingly, the following is expected:

\begin{itemize}
\item Students will make at least six posts to the course blog each semester (i.e., roughly one every two weeks) and posts should be in the neighbourhood of 300-500 words long. The content of the post is to be determined by the student, but it can include a reflection on the readings and discussion. Students are required to make at least six posts per semester, but they can choose to make more contributions than that if they wish: there is no maximum limit, just a minimum requirement. In addition to the six substantive posts, students are also invited to make as many shorter posts containing links to articles or sites of interest as they want. (Note: merely linking to the site or article is not sufficient; students should say why it is interesting and worth looking at.)
\item It is expected that students will actively check the website, read new posts by other students, and comment on those posts.
\end{itemize}

{\large \textsc{Final Project 25\%}}

In small groups (two or three people), students will select a cultural product (film, television show, fiction) and provide an analysis of it demonstrating its relation to course themes. Class time will be allotted to working on the project and students will present their project to the class. Details will be worked out in the second semester, but potential areas could include vampires, zombies, apocalypse and salvation, serial killers, the theory of revenge in rap music. Virtually any recent ``quality'' television series (i.e., produced by AMC, HBO, Starz, etc) is amenable to analysis of this sort, as are many recent films, novels, graphic novels, and video games.

\vspace{.125in}

{\large \textsc{ArtsOne 5\%}}

These marks are allocated to attendance at and participation in Cluster activities, especially the final cluster event in April. Details of the final event will be elaborated during the winter semester.

{\large \textsc{Grades}}

In accordance with the Carleton University Undergraduate Calendar (page 45), the letter grades assigned in this course will have the following percentage equivalents:

\begin{tabbing}
A+ \hspace{.125in} \= 90--100 \hspace{.25in} \= B+ \hspace{.125in} \= 77--79 \hspace{.25in} \= C \hspace{.125in} \= 67--69 \hspace{.25in} \= D+ \hspace{.125in} \= 57--59 \\
A \> 85--89 \> B \> 73--76 \> C \> 63--66 \> D \> 53--56 \\
A- \> 80--84 \> B- \> 70--72 \> C- \> 60--62 \> D- \> 50--52 \\
ABS \> Student absent from final exam \\
DEF \> Deferred \\
FND \> Student could not pass the course even with 100\% on the final exam
\end{tabbing}

For more information on the meaning of grades, see the final page of this syllabus.

{\large \textsc{Please Note}}

This course is a seminar. Seminars differ from lectures in that they pre-suppose active participation from all students and the instructor serves the role of facilitator. Accordingly, it is \textit{expected} that you arrive in class prepared to discuss the assigned material. \textit{If you cannot be bothered to meet this minimum requirement, then you have failed in your basic duty as a student and it is best that you don't bother attending class.} It is your right to do or not do your work as you please, but it is \textit{not} your right to ruin the pedagogical experience of others by failing to be adequately prepared.

In order to facilitate participation---and minimize distraction---computers \emph{will not} be permitted in the classroom (unless the use thereof is an accommodation approved by the Paul Menton Centre or unless a computer is needed to complete in-class assignments). Likewise, texting or any other use of cell phones, iPads, and the like will not be tolerated. If you insist on texting or otherwise fooling around on a ``smartphone,'' tablet or computer, you will be asked to leave. If you'd rather watch YouTube videos or chat on Facebook, you might as well stay home---the seating will be more comfortable and the internet connection better.

It is also expected that students are judicious in their use of email. Hence, when contacting the instructor via email, it is expected that you will use your Carleton account (this is a legal requirement), put the course code and a brief description of the email in the subject line, and write the body of your email in coherent English (i.e., full sentences, proper spelling, grammar and punctuation). If you can't be bothered to write a proper email, I cannot be bothered to reply---after all, the email is clearly not important to you! Questions of general interest or relevance should be posted to cuLearn.

Finally, I cannot emphasize strongly enough how important it is to keep up with assigned readings and to attend all the classes. The material is intentionally difficult and challenging. It is your responsibility to show up ready to learn; it is my job to help you meet this responsibility.

\vspace{.125in}

{\Large \textsc{Academic Regulations, Accommodations, \& Plagiarism}}

University rules regarding registration, withdrawal, appealing marks, and most anything else you might need to know can be found on \href{http://www.carleton.ca/cu0708uc/regulations/acadregsuniv.html}{the university's website}.

{\large \textsc{Requests for Academic Accommodations}}

\textit{For Students with Disabilities} The Paul Menton Centre for Students with Disabilities (PMC) provides services to students with Learning Disabilities (LD), psychiatric/mental health disabilities, Attention Deficit Hyperactivity Disorder (ADHD), Autism Spectrum Disorders (ASD), chronic medical conditions, and impairments in mobility, hearing, and vision. If you have a disability requiring academic accommodations in this course, please contact PMC at 613-520-6608 or \href{mailto:pmc@carleton.ca}{pmc@carleton.ca} for a formal evaluation. If you are already registered with the PMC, contact your PMC coordinator to send me your Letter of Accommodation at the beginning of the term, and no later than two weeks before the first in-class scheduled test or exam requiring accommodation (if applicable). After requesting accommodation from PMC, meet with me to ensure accommodation arrangements are made. Please consult the PMC website for the deadline to request accommodations for the formally-scheduled exam (if applicable).

\textit{For Religious Obligations} Students requesting academic accommodation on the basis of religious obligation should make a formal, written request to their instructors for alternate dates and/or means of satisfying academic requirements. Such requests should be made during the first two weeks of class, or as soon as possible after the need for accommodation is known to exist, but no later than two weeks before the compulsory event. Accommodation is to be worked out directly and on an individual basis between the student and the instructor(s) involved. Instructors will make accommodations in a way that avoids academic disadvantage to the student. Students or instructors who have questions or want to confirm accommodation eligibility of a religious event or practice may refer to the Equity Services website for a list of holy days and Carleton's Academic Accommodation policies, or may contact an Equity Services Advisor in the Equity Services Department for assistance.

\textit{For Pregnancy} Pregnant students requiring academic accommodations are encouraged to contact an Equity Advisor in Equity Services to complete a letter of accommodation. The student must then make an appointment to discuss her needs with the instructor at least two weeks prior to the first academic event in which it is anticipated the accommodation will be required.

\vspace{.125in}

{\large \textsc{Plagiarism}}

Plagiarism is the passing off of someone else's work as your own and is a serious academic offence. For the details of what constitutes plagiarism, the potential penalties and the procedures refer to the section on Instructional Offences in the Undergraduate Calendar.

\textit{What are the Penalties for Plagiarism?} A student found to have plagiarized an assignment may be subject to one of several penalties including: expulsion; suspension from all studies at Carleton; suspension from full-time studies; and/or a reprimand; a refusal of permission to continue or to register in a specific degree program; academic probation; award of an FND, Fail, or an ABS.

\textit{What are the Procedures?} All allegations of plagiarism are reported to the Dean of the Faculty of Arts and Social Sciences. Documentation is prepared by instructors and/or departmental chairs. The Dean writes to the student and the University Ombudsperson about the alleged plagiarism. The Dean reviews the allegation. If it is not resolved at this level then it is referred to a tribunal appointed by the Senate. 

Students are expected to familiarize themselves with and follow the \href{http://www2.carleton.ca/studentaffairs/academic-integrity}{Carleton University Student Academic Integrity Policy}. The Policy is strictly enforced and is binding on all students.

{\large \textsc{Assistance for Students}}

\href{http://www2.carleton.ca/sasc/}{\textbf{Student Academic Success Centre (SASC)} http://www2.carleton.ca/sasc/}

\href{http://www2.carleton.ca/sasc/writing-tutorial-service/}{\textbf{Writing Tutorial Services} http://www2.carleton.ca/sasc/writing-tutorial-service/} 

\href{http://www1.carleton.ca/sasc/peer-assisted-study-sessions/}{\textbf{Peer Assisted Study Sessions (PASS)} http://www1.carleton.ca/sasc/peer-assisted-study-sessions/}

\newpage

\begin{center}
{\Large \textsc{Course Schedule}}

\vspace{.125in}

{\Large \textsc{Fall 2012}}
\end{center}

\vspace{.125in}

\textbf{September 10, 2012}

No assigned readings.

\textbf{September 17, 2012}

\begin{hangparas}{.5in}{1}
Muchembled, Robert. \textit{A History of Violence: From the End of the Middle Ages to the Present.} Translated by Jean Birrell. Cambridge: Polity, 2012. \textbf{Introduction, Chapter 1}

Pinker, Steven. \textit{The Better Angels of Our Nature: Why Violence Has Declined.} New York: Viking, 2011. \textbf{Preface, Chapter 1}
\end{hangparas}

\textbf{September 24, 2012 \hspace{.125in} First Common Session}

Craig McFarlane, ``The Sociology of Violence: Historical and Theoretical Perspectives''

\begin{hangparas}{.5in}{1}
Collins, Randall. \textit{Violence: A Micro-Sociological Theory.} Princeton: Princeton UP, 2008. \textbf{Chapter 1}
\end{hangparas}

\textbf{October 1, 2012}

\begin{hangparas}{.5in}{1}
Muchembled, Robert. \textit{A History of Violence: From the End of the Middle Ages to the Present.} Translated by Jean Birrell. Cambridge: Polity, 2012. \textbf{Chapter 2}

Pinker, Steven. \textit{The Better Angels of Our Nature: Why Violence Has Declined.} New York: Viking, 2011. \textbf{Chapters 2--3}
\end{hangparas}

\textbf{October 8, 2012}

No class

\textbf{October 15, 2012}

\begin{hangparas}{.5in}{1}
Muchembled, Robert. \textit{A History of Violence: From the End of the Middle Ages to the Present.} Translated by Jean Birrell. Cambridge: Polity, 2012. \textbf{Chapters 4--6}
\end{hangparas}

\textbf{October 22, 2012}

\begin{hangparas}{.5in}{1}
Muchembled, Robert. \textit{A History of Violence: From the End of the Middle Ages to the Present.} Translated by Jean Birrell. Cambridge: Polity, 2012. \textbf{Chapters 7--8}
\end{hangparas}

\textbf{October 29, 2012 \hspace{.125in} Second Common Session}

\begin{hangparas}{.5in}{1}
Jillian Crabbe, ``The Rise and Fall of Delinquency''

Reading TBA
\end{hangparas}

\textbf{November 5, 2012}

\begin{hangparas}{.5in}{1}
Pinker, Steven. \textit{The Better Angels of Our Nature: Why Violence Has Declined.} New York: Viking, 2011. \textbf{Chapters 4--5}
\end{hangparas}

\textbf{November 12, 2012}

\begin{hangparas}{.5in}{1}
Pinker, Steven. \textit{The Better Angels of Our Nature: Why Violence Has Declined.} New York: Viking, 2011. \textbf{Chapters 6--7}
\end{hangparas}

\textbf{November 19, 2012}

\begin{hangparas}{.5in}{1}
Muchembled, Robert. \textit{A History of Violence: From the End of the Middle Ages to the Present.} Translated by Jean Birrell. Cambridge: Polity, 2012. \textbf{Chapter 9}

Pinker, Steven. \textit{The Better Angels of Our Nature: Why Violence Has Declined.} New York: Viking, 2011. \textbf{Chapter 8}
\end{hangparas}

\textbf{November 26, 2012}

\begin{hangparas}{.5in}{1}
Pinker, Steven. \textit{The Better Angels of Our Nature: Why Violence Has Declined.} New York: Viking, 2011. \textbf{Chapters 9--10}
\end{hangparas}

\textbf{December 3, 2012 \hspace{.125in} Third Common Session}

Lara Karaian, ``Teenage Sexual Expression or Self-Exploitation? Canada's Extra/Legal Response to Sexting''

\begin{hangparas}{.5in}{1}
Karaian, Lara. ``Lolita Speaks: `Sexting', Teenage Girls and the Law.'' \textit{Crime, Media, Culture} 8, no. 1 (2012): 57-73. \href{http://bit.ly/OJUheG}{\textbf{http://bit.ly/OJUheG}}
\end{hangparas}

\vspace{.125in}

\begin{center}
{\Large \textsc{Winter 2013}}
\end{center}

\vspace{.125in}

\textbf{January 7, 2013}

\begin{hangparas}{.5in}{1}
Kirkman, Robert. \textit{The Walking Dead, Book One.} Berkeley: Image, 2010.

Kee, Chera. ```They are not men\ldots they are dead bodies': From Cannibal to Zombie and Back Again.'' In \textit{Better Off Dead: The Evolution of the Zombie as Post-Human,} edited by Deborah Christie and Sarah Juliet Lauro, 9--23. New York: Fordham UP, 2011. 
\end{hangparas}

\textbf{January 14, 2013}

\begin{hangparas}{.5in}{1}
Shaviro, Steven. ``Gamer.'' In \textit{Post Cinematic Affect,} 93--130. Winchester: Zer0 Books, 2010.

Watch ``Gamer'' (dir. Mark Neveldine and Bryan Taylor, 2009).
\end{hangparas}

\textbf{January 21, 2013}

\begin{hangparas}{.5in}{1}
Wright, Evan. ``The Killer Elite.'' \textit{Rolling Stone} 925 (2003): 56--60, 62, 64--66, 68.\\ \href{http://bit.ly/HTEbKy }{\textbf{http://bit.ly/HTEbKy}}

Wright, Evan. ``From Hell to Baghdad.'' \textit{Rolling Stone} 926 (2003): 52-61.\\ \href{http://bit.ly/HNSVa5}{\textbf{http://bit.ly/HNSVa5}}

Wright, Evan. ``The Battle for Baghdad.'' \textit{Rolling Stone} 927 (2003): 75-80.\\ \href{http://bit.ly/HR8GgP}{\textbf{http://bit.ly/HR8GgP}}

Wright, Evan. ``Dead-Check in Falluja.'' \textit{The Village Voice} November 16, 2004.\\ \href{http://bit.ly/J4aA0X}{\textbf{http://bit.ly/J4aA0X}}

Guest speaker: MCpl. Kyle Dowd (Ret.)
\end{hangparas}

\textbf{January 28, 2013 \hspace{.125in} First Common Session}

\begin{hangparas}{.5in}{1}
Collier, Richard. ``The `Trouble with Boys'? The Child, The Social and the Dangerous Other.'' In \textit{Masculinities, Crime and Criminology: Men, Heterosexuality, and the Criminal(ised) Other,} 67--102. London: Sage, 1998.

Harvey, David. \textit{Rebel Cities: From the Right to the City to the Urban Revolution}. London: Verso, 2012. \textbf{Chapter 6}

Watch ``Attack the Block'' (dir. Joe Cornish, 2011).

Other readings to be announced.
\end{hangparas}

\textbf{February 4 , 2013}

\begin{hangparas}{.5in}{1}
Mi\'eville, China. \textit{The City \& The City.} New York: Del Ray, 2010. \textbf{Part 1}

Durkheim, Emile. \textit{Elementary Forms of Religious Life,} 8--18, 33--9. Translated by Karen E. Fields. New York: The Free Press, 1995.
\end{hangparas}

\newpage

\textbf{February 11, 2013}

\begin{hangparas}{.5in}{1}
Mi\'eville, China. \textit{The City \& The City.} New York: Del Ray, 2010. \textbf{Parts 2, 3, and Coda}

Durkheim, Emile. \textit{Elementary Forms of Religious Life,} 208--16, 303--13, 321--4. Translated by Karen E. Fields. New York: The Free Press, 1995.
\end{hangparas}

\textbf{February 18, 2013}

Reading week.

\textbf{February 25, 2013 \hspace{.125in} Second Common Session}

Guest Lecture TBA

\textbf{March 4, 2013}

\begin{hangparas}{.5in}{1}
Veracini, Lorenzo. ``\textit{District 9} and \textit{Avatar}: Science Fiction and Settler Colonialism.'' \textit{Journal of Intercultural Studies} 32, no. 4 (2011): 355-367. \href{http://bit.ly/Nwpuxx}{\textbf{http://bit.ly/Nwpuxx}}

Watch ``District 9'' (dir. Neil Blomkamp, 2009).
\end{hangparas}

\textbf{March 11 and 18, 2013}

Group work in library.

\textbf{March 25, 2013}

Presentations of drafts of final projects.

\textbf{April 1, 2013 \hspace{.125in} Third Common Session}

Final cluster event.

\newpage

{\footnotesize
\begin{tabular}{|l|l|l|l|}
\hline
\textbf{GPA} & \textbf{\%} & \textbf{Grade} & \textbf{Notes} \\
\hline
\multirow{3}{*}{12} & \multirow{3}{*}{90--100} & \multirow{3}{*}{A+} & \multirow{10}{4in}{\textbf{Excellent.} The assignment demonstrates a thorough understanding of the readings. Comments on materials are perceptive and original and show a comprehension of the material that goes far beyond what is covered in class. Examples are well integrated, relevant, and presented thoughtfully. The essay is \textit{very well written} and free of structural and grammatical errors. The essay is formatted properly and all sources are properly cited. An ``A+'' is very rarely given--this grade is reserved for truly exceptional work. An ``A'' meets \textit{all} these standards. An ``A-'' meets the majority of these standards.} \\ 
& & & \\
& & & \\
\multirow{3}{*}{11} & \multirow{3}{*}{85--89} & \multirow{3}{*}{A} & \\
& & & \\
& & & \\
\multirow{3}{*}{10} & \multirow{3}{*}{80--84} & \multirow{3}{*}{A-} & \\ 
& & & \\
& & & \\
& & & \\
\hline
\multirow{3}{*}{9} & \multirow{3}{*}{77--79} & \multirow{3}{*}{B+} & \multirow{9}{4in}{\textbf{Good.} The assignment demonstrates a good understanding of the readings. Comments on the material are solid without being particularly original. Examples from the material are well integrated and relevant. The is essay \textit{well written} and has only a few structural and grammatical errors. The essay is formatted properly and all sources are properly cited. A ``B+'' meets \textit{all} of these standards. A ``B'' meets a majority of these standards. A ``B-'' meets some of these standards.}\\
& & & \\
& & & \\
\multirow{3}{*}{8} & \multirow{3}{*}{73--76} & \multirow{3}{*}{B} & \\
& & & \\
& & & \\
\multirow{3}{*}{7} & \multirow{3}{*}{70--72} & \multirow{3}{*}{B-} & \\
& & & \\
& & & \\
\hline
\multirow{3}{*}{6} & \multirow{3}{*}{67--69} & \multirow{3}{*}{C+} & \multirow{9}{4in}{\textbf{Adequate.} The assignment presents an adequate comprehension of the material in general, but certain aspects of the material were not clearly understood. Comments on the material are valid but lack insight and originality. Examples from the material are not always convincing and relevant. The essay is competently written but has some structural and grammatical weaknesses, as well as errors in formatting and citations. A ``C+'' meets \textit{all} of these standards. A ``C'' meets a majority of these standards. A ``C-'' meets some of these standards.} \\
& & & \\
& & & \\
\multirow{3}{*}{5} & \multirow{3}{*}{63--66} & \multirow{3}{*}{C} & \\
& & & \\
& & & \\
\multirow{3}{*}{4} & \multirow{3}{*}{60--6} & \multirow{3}{*}{C-} & \\
& & & \\
& & & \\
\hline
\multirow{3}{*}{3} & \multirow{3}{*}{57--59} & \multirow{3}{*}{D+} & \multirow{9}{4in}{\textbf{Poor.} The assignment shows a lack of understanding of many aspects of the materials. Comments on the materials are weak, dealing simplistically with the most obvious aspects of the materials. Examples are not convincing. The essay is not competently written, with several structural and grammatical weaknesses, as well as errors in formatting and citations. Grades in the ``D'' range are near failures. They are used to signal to students that there are \textit{very serious problems} with their work, and that they must greatly improve their performance in order to do adequately in the course.} \\
& & & \\
& & & \\
\multirow{3}{*}{2} & \multirow{3}{*}{53--56} & \multirow{3}{*}{D} & \\
& & & \\
& & & \\
\multirow{3}{*}{1} & \multirow{3}{*}{50--52} & \multirow{3}{*}{D-} & \\
& & & \\
& & & \\
\hline
\multirow{4}{*}{0} & \multirow{4}{*}{0--49} & \multirow{4}{*}{F} & \multirow{4}{4in}{\textbf{Failure.} The essay completely misses the mark and shows no understanding of the material. The essay is incompetently written, full of structural and grammatical problems, as well as errors in formatting and citations.}\\
& & & \\
& & & \\
& & & \\
\hline
\multirow{2}{*}{N/A} & \multirow{2}{*}{N/A} & \multirow{2}{*}{FND} & \multirow{2}{4in}{\textbf{Failure No Deferral.} A major assignment was not submitted and the Registrar has not granted a deferral.}\\
& & & \\
\hline
\multirow{2}{*}{N/A} & \multirow{2}{*}{N/A} & \multirow{2}{*}{GNA} & \multirow{2}{4in}{\textbf{Grade Not Assigned.} Used as a placeholder pending the result of an academic integrity investigation.}\\
& & & \\
\hline
\end{tabular}
}

\end{document}